\begin{center}
\textbf{ABSTRACT}

\vspace{0.5cm}

\textbf{CONTINUUM DEEP-FIELD DETECTION AND \\ ANALYSIS IN SUB-MM AROUND RADIO GALAXIES: \\ DUST AND SYNCHROTRON EMISSION}\\
\vspace{0.5cm}

By\\
\textbf{Nama Mahasiswa\\
NIM: 081320\\
Doctoral Study Program of Astronomy}\\
\end{center}

\vspace{1.0cm}

Radio galaxies constitute a type of \textit{active galaxies} that generate enormous radio power in the range of $10^{34}$ -- $10^{39}$ W. Most of them are hosted by  massive elliptical galaxies which still contain some dense and cold interstellar medium (ISM). Dust emission is mainly detected in the far-infrared (FIR) radiation which is assumed to be thermal and originated from heating processes either by young massive stars or by the active galactic nuclei (AGN). Different mechanisms were proposed to explain the originof these emissions but their respective contributions are not yet conclusive.

Powerful radio-AGN are usually very poor in molecular gas with an average mass of a few $10^8$ M$_\odot$ in the host galaxy. The gas is usually detected as a circumnuclear disk necessary to feed the accretion disk around the central supermassive black hole (SMBH), generating the nuclear activity. In contrast, early-type galaxies -- correspond to elliptical galaxies -- have shown no independent evidence of high star formation rates (SFRs), suggesting that either the older stars or the AGN are responsible for much of the FIR emission. One also argued that in  elliptical galaxies the gas is unrelated to the stellar populations and favor an external origin of the molecular gas. Furthermore, it has been argued that Fanaroff-Riley Type I (FR I) radio galaxies are triggered through a ``dry'' accretion while those of FR II radio galaxies by molecular gas fed towards the center. It has been suggested that ``wet''  minor mergers would be a mechanism to bring molecular gas and dust towards the center which triggers the radio activity.

The radio galaxies (hosted by elliptical galaxies) are frequently found in dense concentration of galaxies, and thus it is crucial to study the ISM properties of the galaxies around these radio galaxies to be able to understand the formation, evolution, and the feedback of the radio galaxy to their environment. To do so, in this dissertation work, we propose to undertake detailed studies in the field of radio galaxies observed by the Atacama Large Millimeter/submillimeter Array (ALMA), which offers unprecedented sensitivity, to investigate particularly the environment around radio galaxies. We will exploit the data acquired from the calibration observations performed for each science project of ALMA. Calibrations are always made, by observing known radio sources that are in vast majority radio galaxies, to set various observational parameters. After selecting some appropriate targets, and by combining the accumulated compatible data, we should be able to reach a sufficiently low noise level at tens $\mu$Jy to obtain deep submillimeter images. We will measure the thermal and/or synchrotron emission in the central radio galaxy and in the field to study the distribution of the ISM and the interplay between the central AGN and its environment. Further analyses should also permit us to disentangle the source of dust heating, AGN or star formation.

\vspace{1.0cm}

\noindent Key words: AGN, radio galaxies, calibrators, continuum, submillimeter



